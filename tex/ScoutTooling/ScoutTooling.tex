% =========================================================================== %
% Scout Tooling
% =========================================================================== %

\ifx\wholebook\relax\else
  \documentclass[a4paper,10pt,twoside]{book}
  %=============================================================================%
% Common things, settings, packages to include
%=============================================================================%

\usepackage{graphicx}
\usepackage{color}
\usepackage{makeidx}
\usepackage{ifpdf}
\usepackage{verbatim}

% --------------------------------------------------------------------------- %
% Setting up stuff depeding on output format
% --------------------------------------------------------------------------- %

\ifpdf
  % special settings for pdf mode
  \usepackage[colorlinks]{hyperref}
  \usepackage{courier}
  
  \hypersetup{
    colorlinks,
    linkcolor=darkblue,
    citecolor=darkblue,
    pdftitle={The Eclipse Scout Book},
    pdfauthor={The Scout Community},
    pdfkeywords={Enterprise Framework, Eclipse, Java, Client-Side, Rich Client, Web Client, Mobile},
    pdfsubject={Computer Science}
  }
  
  \usepackage{caption}
  \captionsetup{margin=10pt,font=small,labelfont=bf}
\else
  % special stuff for html mode
  \usepackage[tex4ht]{hyperref}
\fi

% --------------------------------------------------------------------------- %
% Setting up printing range
% --------------------------------------------------------------------------- %

\parindent 1cm
\parskip 0.2cm
\topmargin 0.2cm
\oddsidemargin 1cm
\evensidemargin 0.5cm
\textwidth 15cm
\textheight 21cm

% --------------------------------------------------------------------------- %
% Setting up listings
% --------------------------------------------------------------------------- %

\usepackage{listings}
 
\definecolor{darkviolet}{rgb}{0.5,0,0.4}
\definecolor{darkgreen}{rgb}{0,0.4,0.2} 
\definecolor{darkblue}{rgb}{0.1,0.1,0.9}
\definecolor{darkgrey}{rgb}{0.5,0.5,0.5}
\definecolor{lightblue}{rgb}{0.4,0.4,1}
\definecolor{lightgray}{rgb}{0.97,0.97,0.97}

\renewcommand{\lstlistlistingname}{List of Listings}

% general settings
\lstset{
  basicstyle=\small\ttfamily,
  columns=fullflexible,
  breaklines=true,
  breakindent=10pt,
  prebreak=\mbox{{\color{blue}\tiny$\searrow$}},
  postbreak=\mbox{{\color{blue}\tiny$\rightarrow$}},
  showstringspaces=false,
  backgroundcolor=\color{lightgray}
}

% settings for xml files
\lstdefinelanguage{xml}
{
  commentstyle=\color{darkgrey}\upshape,
  morestring=[b]",
  morestring=[s]{>}{<},
  morecomment=[s]{<?}{?>},
  stringstyle=\color{black},
  identifierstyle=\color{darkblue},
  keywordstyle=\color{cyan},
  morekeywords={xmlns,name,point,factory,class}% list your attributes here
}

% settings for ini files
\lstdefinelanguage{ini}
{
  morecomment=[f][\color{darkgrey}\upshape][0]\#, % # is comment iff it's the first char on the line
  stringstyle=\color{black}
}

% default settings (for java files)
\lstset{
  language=Java,
  emphstyle=\color{red}\bfseries,
  keywordstyle=\color{darkviolet}\bfseries,
  commentstyle=\color{darkgreen},
  morecomment=[s][\color{lightblue}]{/**}{*/},
  stringstyle=\color{darkblue},
}

% --------------------------------------------------------------------------- %
% cross reference macros
% --------------------------------------------------------------------------- %
\newcommand{\applabel}[1]{\label{apx:#1}}
\newcommand{\chalabel}[1]{\label{cha:#1}}
\newcommand{\seclabel}[1]{\label{sec:#1}}
\newcommand{\lstlabel}[1]{\label{lst:#1}}
\newcommand{\figlabel}[1]{\label{fig:#1}}
\newcommand{\tablabel}[1]{\label{tab:#1}}

\newcommand{\appref}[1]{Appendix~\ref{apx:#1}}
\newcommand{\charef}[1]{Chapter~\ref{cha:#1}\xspace}
\newcommand{\secref}[1]{Section~\ref{sec:#1}}
\newcommand{\lstref}[1]{Listing~\ref{lst:#1}\xspace}
\newcommand{\figref}[1]{Figure~\ref{fig:#1}\xspace}
\newcommand{\tabref}[1]{Table~\ref{tab:#1}\xspace}

% --------------------------------------------------------------------------- %
% graphics paths
% --------------------------------------------------------------------------- %
\graphicspath{
  {figures/}
  {Introduction/figures/}
}

%=============================================================================%

  \pagestyle{headings}
  \graphicspath{{figures/} {../figures/}}
  \begin{document}
  \sloppy
\fi

% --------------------------------------------------------------------------- %
\chapter{Scout Tooling}
\chalabel{tooling}

In addition to the Scout runtime framework presented in the previous chapter, Eclipse Scout also includes a comprehensive tooling, the Scout SDK. 
Thanks to this tooling, developing Scout applications is made simpler, more productive and also more robust. 
Initially, a solid understanding of the Java language is sufficient to start developing Scout applications and only a rough understanding of the underlying Eclipse/OSGi/J2EE technologies is required. 

The Scout SDK consists of navigation support for the application model defined by the Scout runtime and provides many intuitive component wizards. 
This creates an ideal environment to beginners for building complete, high-quality Scout applications. 
Typically, Java developers only need a few days of Scout training to start creating their own advanced client server business applications. 

The Scout SDK also helps developers to become more productive.
Many repetitive and error prone tasks run automatically in the background or are taken care of by the component wizards of the Scout SDK. 
This starts with the initial creation of a Scout client server application, continues with the wizards to create complete dialogs and pages and includes the automatic management of the data transfer objects needed by the client server communication.

Finally, the application code created by the Scout SDK wizards helps to ensure that the resulting Scout application has a consistent and robust code base and is well aligned with the application model defined by the Scout runtime framework.

% --------------------------------------------------------------------------- %
\section{The Scout SDK}

The Scout SDK contains the Scout Explorer, the Scout Objects Properties view, and a wide range of wizards. The
user navigates through the application model in the Scout
Explorer. The Scout Objects Properties view provides direct access to the available Scout properties for the selected element in the Scout Explorer. 
The wizards support the creation of application components, such as dialogues
on the client side or services on the server side by generating the necessary Java code.

The Scout SDK provides round-trip-engineering for the application model and the Java source files. Each developer
in a team is free to decide individually whether to work in the Scout perspective, the Java perspective or to switch
back and forth between the two.

\noindent Existing Documentation
\begin{itemize}
  \item concept wiki \url{http://wiki.eclipse.org/Scout/SDK}
\end{itemize}

% --------------------------------------------------------------------------- %
\section{The Scout Perspective}
needs text

\noindent Existing Documentation
\begin{itemize}
  \item concept wiki \url{http://wiki.eclipse.org/Scout/SDK/Perspective}
\end{itemize}

\subsection{Other Userful IDE Perspectives}
needs text

- java perspective
- debug perspective

links to java ide stuff

% --------------------------------------------------------------------------- %
\section{The Scout Explorer}
needs text

\noindent Existing Documentation
\begin{itemize}
  \item concept wiki: \url{http://wiki.eclipse.org/Scout/SDK/Explorer_View}
\end{itemize}

% --------------------------------------------------------------------------- %
\section{Scout Object Properties}
needs text

\noindent Existing Documentation
\begin{itemize}
  \item concept wiki: \url{http://wiki.eclipse.org/Scout/SDK/Object_Properties_View}
\end{itemize}

% --------------------------------------------------------------------------- %
\section{Scout Wizards}
needs text

% --------------------------------------------------------------------------- %
\subsection{Creating a new Scout Project}
needs text

\noindent Existing Documentation
\begin{itemize}
  \item how-to wiki \url{http://wiki.eclipse.org/Scout/HowTo/3.8/Create_a_new_project}
\end{itemize}

What is a Scout Project?
needs text

\noindent Existing Documentation
\begin{itemize}
  \item forum \url{http://www.eclipse.org/forums/index.php/t/395379/}
  \item wiki concept \url{http://wiki.eclipse.org/Scout/Concepts#Scout_Project}
\end{itemize}

% --------------------------------------------------------------------------- %
\subsection{Creating a Form}
needs text

New Form Field Wizard

% --------------------------------------------------------------------------- %
\subsection{Creating a Search Form}
needs text

New Form Field Wizard

% --------------------------------------------------------------------------- %
\subsection{Creating a Form Field}
needs text

New Form Field Wizard

% --------------------------------------------------------------------------- %
\subsection{Creating an Outline}
needs text

New Form Field Wizard

% --------------------------------------------------------------------------- %
\subsection{Creating a Page}
needs text

New Form Field Wizard


% --------------------------------------------------------------------------- %
\subsection{Creating a Library Bundle}
needs text

New library bundle wizard

% --------------------------------------------------------------------------- %
\subsection{Creating a Permission}
needs text

New permission wizard

% --------------------------------------------------------------------------- %
\subsection{Creating Code Types and Codes}
needs text

New code type and new code wizards

% --------------------------------------------------------------------------- %
\subsection{Creating a Server Service and Service Operations}
needs text

New code type and new code wizards

% --------------------------------------------------------------------------- %
\subsection{Exporting a Scout Project}
needs text

\noindent Existing Documentation
\begin{itemize}
  \item forum: access to logs \url{http://www.eclipse.org/forums/index.php/t/367447/}
\end{itemize}

\ifx\wholebook\relax\else
   \begin{thebibliography}{99}
  \addcontentsline{toc}{chapter}{Bibliography}
  
  % add/insert books in alphabetical order of 1st author
  
  \bibitem{batessierra05}
    \textit{Bert Bates, Kathy Sierra},
	\textbf{Head First Java} 2nd edition, 
	O'Reilly Media, 2005.

  \bibitem{bloch08} 
    \textit{Joshua Bloch},
    \textbf{Effective Java} 2nd edition, 
	Addison-Wesley, 2008.
	
  \bibitem{eckel06}
    \textit{Bruce Eckel},
	\textbf{Thinking in Java} 4th edition, 
	Prentice Hall International, 2006.

\end{thebibliography}

   \end{document}
\fi

% =========================================================================== %
