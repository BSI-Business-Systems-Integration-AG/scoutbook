% =========================================================================== %
% Scout Runtime
% =========================================================================== %

\ifx\wholebook\relax\else
  \documentclass[a4paper,10pt,twoside]{book}
  %=============================================================================%
% Common things, settings, packages to include
%=============================================================================%

\usepackage{graphicx}
\usepackage{color}
\usepackage{makeidx}
\usepackage{ifpdf}
\usepackage{verbatim}

% --------------------------------------------------------------------------- %
% Setting up stuff depeding on output format
% --------------------------------------------------------------------------- %

\ifpdf
  % special settings for pdf mode
  \usepackage[colorlinks]{hyperref}
  \usepackage{courier}
  
  \hypersetup{
    colorlinks,
    linkcolor=darkblue,
    citecolor=darkblue,
    pdftitle={The Eclipse Scout Book},
    pdfauthor={The Scout Community},
    pdfkeywords={Enterprise Framework, Eclipse, Java, Client-Side, Rich Client, Web Client, Mobile},
    pdfsubject={Computer Science}
  }
  
  \usepackage{caption}
  \captionsetup{margin=10pt,font=small,labelfont=bf}
\else
  % special stuff for html mode
  \usepackage[tex4ht]{hyperref}
\fi

% --------------------------------------------------------------------------- %
% Setting up printing range
% --------------------------------------------------------------------------- %

\parindent 1cm
\parskip 0.2cm
\topmargin 0.2cm
\oddsidemargin 1cm
\evensidemargin 0.5cm
\textwidth 15cm
\textheight 21cm

% --------------------------------------------------------------------------- %
% Setting up listings
% --------------------------------------------------------------------------- %

\usepackage{listings}
 
\definecolor{darkviolet}{rgb}{0.5,0,0.4}
\definecolor{darkgreen}{rgb}{0,0.4,0.2} 
\definecolor{darkblue}{rgb}{0.1,0.1,0.9}
\definecolor{darkgrey}{rgb}{0.5,0.5,0.5}
\definecolor{lightblue}{rgb}{0.4,0.4,1}
\definecolor{lightgray}{rgb}{0.97,0.97,0.97}

\renewcommand{\lstlistlistingname}{List of Listings}

% general settings
\lstset{
  basicstyle=\small\ttfamily,
  columns=fullflexible,
  breaklines=true,
  breakindent=10pt,
  prebreak=\mbox{{\color{blue}\tiny$\searrow$}},
  postbreak=\mbox{{\color{blue}\tiny$\rightarrow$}},
  showstringspaces=false,
  backgroundcolor=\color{lightgray}
}

% settings for xml files
\lstdefinelanguage{xml}
{
  commentstyle=\color{darkgrey}\upshape,
  morestring=[b]",
  morestring=[s]{>}{<},
  morecomment=[s]{<?}{?>},
  stringstyle=\color{black},
  identifierstyle=\color{darkblue},
  keywordstyle=\color{cyan},
  morekeywords={xmlns,name,point,factory,class}% list your attributes here
}

% settings for ini files
\lstdefinelanguage{ini}
{
  morecomment=[f][\color{darkgrey}\upshape][0]\#, % # is comment iff it's the first char on the line
  stringstyle=\color{black}
}

% default settings (for java files)
\lstset{
  language=Java,
  emphstyle=\color{red}\bfseries,
  keywordstyle=\color{darkviolet}\bfseries,
  commentstyle=\color{darkgreen},
  morecomment=[s][\color{lightblue}]{/**}{*/},
  stringstyle=\color{darkblue},
}

% --------------------------------------------------------------------------- %
% cross reference macros
% --------------------------------------------------------------------------- %
\newcommand{\applabel}[1]{\label{apx:#1}}
\newcommand{\chalabel}[1]{\label{cha:#1}}
\newcommand{\seclabel}[1]{\label{sec:#1}}
\newcommand{\lstlabel}[1]{\label{lst:#1}}
\newcommand{\figlabel}[1]{\label{fig:#1}}
\newcommand{\tablabel}[1]{\label{tab:#1}}

\newcommand{\appref}[1]{Appendix~\ref{apx:#1}}
\newcommand{\charef}[1]{Chapter~\ref{cha:#1}\xspace}
\newcommand{\secref}[1]{Section~\ref{sec:#1}}
\newcommand{\lstref}[1]{Listing~\ref{lst:#1}\xspace}
\newcommand{\figref}[1]{Figure~\ref{fig:#1}\xspace}
\newcommand{\tabref}[1]{Table~\ref{tab:#1}\xspace}

% --------------------------------------------------------------------------- %
% graphics paths
% --------------------------------------------------------------------------- %
\graphicspath{
  {figures/}
  {Introduction/figures/}
}

%=============================================================================%

  \pagestyle{headings}
  \graphicspath{{figures/} {../figures/}}
  \begin{document}
  \sloppy
\fi


% --------------------------------------------------------------------------- %
\chapter{Scout Runtime}
\chalabel{runtime}

One of the primary goals of the Scout runtime framework is to let the application developer focus on implementing business requirements. 
To support this goal, Scout covers most recurring aspects of business applications from the user interface to the integration of backend services. 
To clearly state the boundary of the scope of the framework, the integration of Scout application in a typical enterprise setup is discussed in the following section.

\secref{application_architecture} describes the architecture of typical Scout applications including both the server and clients. 


% --------------------------------------------------------------------------- %
\section{Enterprise Integration}

\begin{figure}
\includediagram{14cm}{scout_integration}
\caption{The recommended usage of the Scout framework for enterprise applications.}
\figlabel{scout_integration_enterprise}
\end{figure}

The scope of the Scout framework is best explained in the context of an enterprise setup as shown in \figref{scout_integration_enterprise}. 
From

not persistence, not business entity modeling, not ?
but user interaction handling, business rules, accessing existing services offered through enterprise service bus, transparent client server communication

% --------------------------------------------------------------------------- %
\section{Application Architecture}
\seclabel{application_architecture}
needs text

\noindent Existing Documentation
\begin{itemize}
  \item wiki \url{http://wiki.eclipse.org/Scout/Concepts}
\end{itemize}

\begin{figure}
\includediagram{14cm}{scout_app_architecture_desktop}
\caption{The architecture of a typical Scout client server application.
On the left side, a Scout desktop client is depicted.}
\figlabel{scout_app_architecture_desktop}
\end{figure}

Architecture including desktop clients.

\begin{figure}
\includediagram{14cm}{scout_app_architecture_web}
\caption{A typical Scout client server setup including web and mobile clients.}
\figlabel{scout_app_architecture_web}
\end{figure}

Architecture when wordking with Ajax server for web applications

% --------------------------------------------------------------------------- %
\section{Multi-Frontend Support}

two important aspects: 1) same app running on many devices 2) effective risk reduction strategy: no mixing of business logic an ui tech code, deciding for the 'wrong' framwork not so bad any more

% --------------------------------------------------------------------------- %
\section{The Scout Client}
needs text

 large collection of mature UI components.
 Scout supports various UI technologies out of the box. 
 Depending on your needs, you decide to build applications for
mobile devices, the browser or the desktop. Mobile and
web applications are based on Eclipse RAP. Desktop clients
are based on either Swing or SWT.

internationalization, link to packaging with client, shared, ui tech, framwork + other bundles. 

% --------------------------------------------------------------------------- %
\section{The Scout Server}

For seamless integration into a service-oriented IT architecture, Scout offers direct support for Web services based on
JAX-WS. To access relational databases other than Apache
Derby, connectors are also available for non-EPL compatible databases such as Oracle, MySql, PostgreSQL or DB2

link to packaging with server, shared, framwork + other bundles

% --------------------------------------------------------------------------- %
\section{Client Server Communication}
needs text

service tunnel

\noindent Existing Documentation
\begin{itemize}
  \item forum tech background questions \url{http://www.eclipse.org/forums/index.php/t/299623/}
  \item concept wiki \url{http://wiki.eclipse.org/Scout/Concepts/Communication}
\end{itemize}

same pointer in onedaytutorial and server


\ifx\wholebook\relax\else
   \begin{thebibliography}{99}
  \addcontentsline{toc}{chapter}{Bibliography}
  
  % add/insert books in alphabetical order of 1st author
  
  \bibitem{batessierra05}
    \textit{Bert Bates, Kathy Sierra},
	\textbf{Head First Java} 2nd edition, 
	O'Reilly Media, 2005.

  \bibitem{bloch08} 
    \textit{Joshua Bloch},
    \textbf{Effective Java} 2nd edition, 
	Addison-Wesley, 2008.
	
  \bibitem{eckel06}
    \textit{Bruce Eckel},
	\textbf{Thinking in Java} 4th edition, 
	Prentice Hall International, 2006.

\end{thebibliography}

   \end{document}
\fi

% =========================================================================== %
