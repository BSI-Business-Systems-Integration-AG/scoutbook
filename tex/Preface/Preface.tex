% =========================================================================== %
% Preface
% please make sure that this fits into two pages (max)
% =========================================================================== %

\ifx\wholebook\relax\else
  \documentclass[a4paper,10pt,twoside]{book}
  %=============================================================================%
% Common things, settings, packages to include
%=============================================================================%

\usepackage{graphicx}
\usepackage{color}
\usepackage{makeidx}
\usepackage{ifpdf}
\usepackage{verbatim}

% --------------------------------------------------------------------------- %
% Setting up stuff depeding on output format
% --------------------------------------------------------------------------- %

\ifpdf
  % special settings for pdf mode
  \usepackage[colorlinks]{hyperref}
  \usepackage{courier}
  
  \hypersetup{
    colorlinks,
    linkcolor=darkblue,
    citecolor=darkblue,
    pdftitle={The Eclipse Scout Book},
    pdfauthor={The Scout Community},
    pdfkeywords={Enterprise Framework, Eclipse, Java, Client-Side, Rich Client, Web Client, Mobile},
    pdfsubject={Computer Science}
  }
  
  \usepackage{caption}
  \captionsetup{margin=10pt,font=small,labelfont=bf}
\else
  % special stuff for html mode
  \usepackage[tex4ht]{hyperref}
\fi

% --------------------------------------------------------------------------- %
% Setting up printing range
% --------------------------------------------------------------------------- %

\parindent 1cm
\parskip 0.2cm
\topmargin 0.2cm
\oddsidemargin 1cm
\evensidemargin 0.5cm
\textwidth 15cm
\textheight 21cm

% --------------------------------------------------------------------------- %
% Setting up listings
% --------------------------------------------------------------------------- %

\usepackage{listings}
 
\definecolor{darkviolet}{rgb}{0.5,0,0.4}
\definecolor{darkgreen}{rgb}{0,0.4,0.2} 
\definecolor{darkblue}{rgb}{0.1,0.1,0.9}
\definecolor{darkgrey}{rgb}{0.5,0.5,0.5}
\definecolor{lightblue}{rgb}{0.4,0.4,1}
\definecolor{lightgray}{rgb}{0.97,0.97,0.97}

\renewcommand{\lstlistlistingname}{List of Listings}

% general settings
\lstset{
  basicstyle=\small\ttfamily,
  columns=fullflexible,
  breaklines=true,
  breakindent=10pt,
  prebreak=\mbox{{\color{blue}\tiny$\searrow$}},
  postbreak=\mbox{{\color{blue}\tiny$\rightarrow$}},
  showstringspaces=false,
  backgroundcolor=\color{lightgray}
}

% settings for xml files
\lstdefinelanguage{xml}
{
  commentstyle=\color{darkgrey}\upshape,
  morestring=[b]",
  morestring=[s]{>}{<},
  morecomment=[s]{<?}{?>},
  stringstyle=\color{black},
  identifierstyle=\color{darkblue},
  keywordstyle=\color{cyan},
  morekeywords={xmlns,name,point,factory,class}% list your attributes here
}

% settings for ini files
\lstdefinelanguage{ini}
{
  morecomment=[f][\color{darkgrey}\upshape][0]\#, % # is comment iff it's the first char on the line
  stringstyle=\color{black}
}

% default settings (for java files)
\lstset{
  language=Java,
  emphstyle=\color{red}\bfseries,
  keywordstyle=\color{darkviolet}\bfseries,
  commentstyle=\color{darkgreen},
  morecomment=[s][\color{lightblue}]{/**}{*/},
  stringstyle=\color{darkblue},
}

% --------------------------------------------------------------------------- %
% cross reference macros
% --------------------------------------------------------------------------- %
\newcommand{\applabel}[1]{\label{apx:#1}}
\newcommand{\chalabel}[1]{\label{cha:#1}}
\newcommand{\seclabel}[1]{\label{sec:#1}}
\newcommand{\lstlabel}[1]{\label{lst:#1}}
\newcommand{\figlabel}[1]{\label{fig:#1}}
\newcommand{\tablabel}[1]{\label{tab:#1}}

\newcommand{\appref}[1]{Appendix~\ref{apx:#1}}
\newcommand{\charef}[1]{Chapter~\ref{cha:#1}\xspace}
\newcommand{\secref}[1]{Section~\ref{sec:#1}}
\newcommand{\lstref}[1]{Listing~\ref{lst:#1}\xspace}
\newcommand{\figref}[1]{Figure~\ref{fig:#1}\xspace}
\newcommand{\tabref}[1]{Table~\ref{tab:#1}\xspace}

% --------------------------------------------------------------------------- %
% graphics paths
% --------------------------------------------------------------------------- %
\graphicspath{
  {figures/}
  {Introduction/figures/}
}

%=============================================================================%

  \pagestyle{headings}
  \graphicspath{{figures/} {../figures/}}
  \begin{document}
  \sloppy
\fi


% --------------------------------------------------------------------------- %
\chapter{Preface}


% --------------------------------------------------------------------------- %
\section*{Who should read this Book?}

text needed

\begin{itemize}
\item You want to build a multi-user mobile application.
\item You have to build Java based business application. 
\item You need to decide on the IT strategy of your organisation
\item You would like to show off the great library you have built but have hardly time for the good looking UI that it deserves
\end{itemize}

If any of the above scenarios is relevant in your situation, you might want
to consider \secref{whatshouldiread} on what to read and in which order.

% --------------------------------------------------------------------------- %
\section*{This Book is Open and Free}

\begin{itemize}
\item Access is simple. To download the book you do not need to register, 
look at ads or pay subscription fees. To cater for individual
preferences the book is avilable in various formats. 
Eventually, there should also be a printed version. And translations.

\item Reuse/rehash is simple. This book is released under the Creative Commons (CC-BY)
license. See~\appref{cclicence} for details.

\item Contribution is simple. That's why the complete sources is available on github
and we maintain a continuous integration (CI) infrastucture for building the book. 
The necessary setup for your own CI setup is described in the Scout Wiki: \\
\url{http://wiki.eclipse.org/Scout/Book/}
\end{itemize}

In short, you are allowed to freely use, share and adapt this book, as long as you 
respect the conditions of the licenses mentioned above.


% --------------------------------------------------------------------------- %
\section*{The Scout Community}

The community around Scout is its most valuable asset. That's why it deserves 
beeing mentionend as early as possible. Well, after you have decided that 
you possibly belong to the target audience of this book.

The Scout community is growing, friendly and active.
Here is a short list of resources that you may find useful:

\begin{itemize}
\item \url{http://www.eclipse.org/scout} Web site.
\item \url{http://wiki.eclipse.org/scout} Wiki pages.
\item \url{http://www.eclipse.org/forums/eclipse.scout} The Forum.
\item \url{http://www.bsiag.com/scout} Scout blogging.
\item \url{http://twitter.com/EclipseScout} Scout tweets.
\end{itemize}

% --------------------------------------------------------------------------- %
\section*{Thank You!}

We thank the various people and organisations that have contributed 
to this book. In particular, we thank xxx
Many good ideas regarding various aspects, such as book setup, etc. taken from 
numerous freely available resources on the web. 
Significant impact had ''Pharo by Example'', ...

We thank BSI Business Systems Integration AG for having made the bold move 
to open source the Eclipse Scout framework in the first place and its 
dedication to open standards in general and its continued support to further
evolve Scout and its growing community.

Last but not least we thank the Scout community for the many corrections and improvements 
made during the creation of the first version of The Eclipse Scout Book.

% --------------------------------------------------------------------------- %

\ifx\wholebook\relax\else
   \begin{thebibliography}{99}
  \addcontentsline{toc}{chapter}{Bibliography}
  
  % add/insert books in alphabetical order of 1st author
  
  \bibitem{batessierra05}
    \textit{Bert Bates, Kathy Sierra},
	\textbf{Head First Java} 2nd edition, 
	O'Reilly Media, 2005.

  \bibitem{bloch08} 
    \textit{Joshua Bloch},
    \textbf{Effective Java} 2nd edition, 
	Addison-Wesley, 2008.
	
  \bibitem{eckel06}
    \textit{Bruce Eckel},
	\textbf{Thinking in Java} 4th edition, 
	Prentice Hall International, 2006.

\end{thebibliography}

   \end{document}
\fi

% =========================================================================== %
