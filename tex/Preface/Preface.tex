% =========================================================================== %
% Preface
% please make sure that this fits into two pages (max)
% =========================================================================== %

\ifx\wholebook\relax\else
  \documentclass[a4paper,10pt,twoside]{book}
  %=============================================================================%
% Common things, settings, packages to include
%=============================================================================%

\usepackage{graphicx}
\usepackage{color}
\usepackage{makeidx}
\usepackage{ifpdf}
\usepackage{verbatim}

% --------------------------------------------------------------------------- %
% Setting up stuff depeding on output format
% --------------------------------------------------------------------------- %

\ifpdf
  % special settings for pdf mode
  \usepackage[colorlinks]{hyperref}
  \usepackage{courier}
  
  \hypersetup{
    colorlinks,
    linkcolor=darkblue,
    citecolor=darkblue,
    pdftitle={The Eclipse Scout Book},
    pdfauthor={The Scout Community},
    pdfkeywords={Enterprise Framework, Eclipse, Java, Client-Side, Rich Client, Web Client, Mobile},
    pdfsubject={Computer Science}
  }
  
  \usepackage{caption}
  \captionsetup{margin=10pt,font=small,labelfont=bf}
\else
  % special stuff for html mode
  \usepackage[tex4ht]{hyperref}
\fi

% --------------------------------------------------------------------------- %
% Setting up printing range
% --------------------------------------------------------------------------- %

\parindent 1cm
\parskip 0.2cm
\topmargin 0.2cm
\oddsidemargin 1cm
\evensidemargin 0.5cm
\textwidth 15cm
\textheight 21cm

% --------------------------------------------------------------------------- %
% Setting up listings
% --------------------------------------------------------------------------- %

\usepackage{listings}
 
\definecolor{darkviolet}{rgb}{0.5,0,0.4}
\definecolor{darkgreen}{rgb}{0,0.4,0.2} 
\definecolor{darkblue}{rgb}{0.1,0.1,0.9}
\definecolor{darkgrey}{rgb}{0.5,0.5,0.5}
\definecolor{lightblue}{rgb}{0.4,0.4,1}
\definecolor{lightgray}{rgb}{0.97,0.97,0.97}

\renewcommand{\lstlistlistingname}{List of Listings}

% general settings
\lstset{
  basicstyle=\small\ttfamily,
  columns=fullflexible,
  breaklines=true,
  breakindent=10pt,
  prebreak=\mbox{{\color{blue}\tiny$\searrow$}},
  postbreak=\mbox{{\color{blue}\tiny$\rightarrow$}},
  showstringspaces=false,
  backgroundcolor=\color{lightgray}
}

% settings for xml files
\lstdefinelanguage{xml}
{
  commentstyle=\color{darkgrey}\upshape,
  morestring=[b]",
  morestring=[s]{>}{<},
  morecomment=[s]{<?}{?>},
  stringstyle=\color{black},
  identifierstyle=\color{darkblue},
  keywordstyle=\color{cyan},
  morekeywords={xmlns,name,point,factory,class}% list your attributes here
}

% settings for ini files
\lstdefinelanguage{ini}
{
  morecomment=[f][\color{darkgrey}\upshape][0]\#, % # is comment iff it's the first char on the line
  stringstyle=\color{black}
}

% default settings (for java files)
\lstset{
  language=Java,
  emphstyle=\color{red}\bfseries,
  keywordstyle=\color{darkviolet}\bfseries,
  commentstyle=\color{darkgreen},
  morecomment=[s][\color{lightblue}]{/**}{*/},
  stringstyle=\color{darkblue},
}

% --------------------------------------------------------------------------- %
% cross reference macros
% --------------------------------------------------------------------------- %
\newcommand{\applabel}[1]{\label{apx:#1}}
\newcommand{\chalabel}[1]{\label{cha:#1}}
\newcommand{\seclabel}[1]{\label{sec:#1}}
\newcommand{\lstlabel}[1]{\label{lst:#1}}
\newcommand{\figlabel}[1]{\label{fig:#1}}
\newcommand{\tablabel}[1]{\label{tab:#1}}

\newcommand{\appref}[1]{Appendix~\ref{apx:#1}}
\newcommand{\charef}[1]{Chapter~\ref{cha:#1}\xspace}
\newcommand{\secref}[1]{Section~\ref{sec:#1}}
\newcommand{\lstref}[1]{Listing~\ref{lst:#1}\xspace}
\newcommand{\figref}[1]{Figure~\ref{fig:#1}\xspace}
\newcommand{\tabref}[1]{Table~\ref{tab:#1}\xspace}

% --------------------------------------------------------------------------- %
% graphics paths
% --------------------------------------------------------------------------- %
\graphicspath{
  {figures/}
  {Introduction/figures/}
}

%=============================================================================%

	\pagestyle{headings}
  \graphicspath{{figures/} {../figures/}}
  \begin{document}
  \sloppy
\fi


% --------------------------------------------------------------------------- %
\chapter{Preface}

Today, the Java platform is widely seen as the primary choice for implementing enterprise applications. 
While many successful frameworks support the development of persistence layers and business services, implementing front-ends in a simple and clean way remains a challenge. 
This is exactly where Eclipse Scout fits in. 
The primary goal of Scout is to make your life as a developer easier and to help organisations to save money and time. 
For this, the Scout framework covers most of the recurring front-end aspects such as user authentication, client-server communication and the user interface. 
This comprehensive scope reduces the amount of necessary boiler plate code, and let developers concentrate on understanding and implementing business functionality. 

The purpose of this book is to get the reader familiar with the Scout framework.
In this book Scout's core features are introduced and explained using many practical examples. 
And as both the Scout framework and Scout applications are written in Java, we make the assumption that you are familiar with the language too. 
Ideally, you have worked with Java for some time now and feel comfortable with the basic language features. 

In the first part of the book a general introduction into the runtime part of the framework and the tooling - the Scout SDK - is provided. 
After the mandatory ''Hello World!'' application, the book walks you though a complete client server application including database access. 
The focus of the book's second part is on the front-end side of Scout applications. 
First, an overview of the Scout client model is introduced before Scout's most important UI components are described based on the Scout widget demo application. 
To cover the the server-side of Scout applications, an additional part of the book is planned to be released jointly with version 5.0 of the Scout framework. 
And finally, we intend to amend the book regarding building, testing and continuous integration for Scout applications. 

Last but not least, we thank you for your interest in Scout, for being part of our community and for your friendly support of new community members.
To allow for contributions to this book, the technical setup and the book's licence have been selected to minimize restrictions. 
According to the terms of the Creative Commons (CC-BY) license, you are allowed to freely use, share and adapt this book. 
All source files of the book including the Scout projects described in the book are available on github.
For the first edition of this book, we did already receive a number of bug reports and comments that were pointing out mistakes, inconsistencies and suggestions for changes. 
This feedback is very valuable to us as it helps to improve both the book's content and the quality for all future readers. 
We hope that this book helps you to get started quickly and would love to get your feedback. 

% --------------------------------------------------------------------------- %

\ifx\wholebook\relax\else
   \begin{thebibliography}{99}
  \addcontentsline{toc}{chapter}{Bibliography}
  
  % add/insert books in alphabetical order of 1st author
  
  \bibitem{batessierra05}
    \textit{Bert Bates, Kathy Sierra},
	\textbf{Head First Java} 2nd edition, 
	O'Reilly Media, 2005.

  \bibitem{bloch08} 
    \textit{Joshua Bloch},
    \textbf{Effective Java} 2nd edition, 
	Addison-Wesley, 2008.
	
  \bibitem{eckel06}
    \textit{Bruce Eckel},
	\textbf{Thinking in Java} 4th edition, 
	Prentice Hall International, 2006.

\end{thebibliography}

   \end{document}
\fi

% =========================================================================== %
